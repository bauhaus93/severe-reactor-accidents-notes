\documentclass[12pt]{article}

\usepackage[cm]{fullpage}
\usepackage{amsthm}
\usepackage{amsmath}
\usepackage{amsfonts}
\usepackage{amssymb}
\usepackage{xspace}
\usepackage[german]{babel}
\usepackage{fancyhdr}
\usepackage{titling}
\usepackage{textcomp}
\usepackage{subcaption}
\usepackage{enumitem}

\begin{document}

\thispagestyle{empty}

\section{Reaktortypen/Sicherheitskonzepte}

\subsection{Siedewasserrekator}

\subsubsection{Allgemein}
\begin{itemize}[noitemsep]
	\item Kühlmittel: Wasser
	\item Moderator: Wasser
	\item Absorber: Borcarbid (\(B_4C\))
	\item Brennstoff: Angereichertes \(UO_2\)
	\item Wasser kocht im Kern
	\item Steuerstäbe von unten
	\item Dampf direkt von Kern zu Turbine (nach Dampf/Wasser Separator)
	\item Druck: 7 MPa
	\item Temperatur: 300°C
	\item Thermische Leistung: 2.1 GW
	\item Effizenz: 34 \%
\end{itemize}

\subsubsection{Sicherheit}
\begin{itemize}[noitemsep]
	\item Negativer Void-Koeffizent (Mehr Dampf \textrightarrow\ Weniger Reaktivität)
	\item 2 unabhängige Abschaltsysteme
		\begin{itemize}[noitemsep]
			\item Steuerstäbe einfahren (Hydraulisches System, Steuerstäbe sind ja unten)
			\item Einspeisung Borsalzlösung
		\end{itemize}
	\item Redundante Sicherheitssysteme
	\item Physisch separierte Sicherheitssysteme
	\item 5 Barrieren zwischen Spaltprodukten/Umwelt
\end{itemize}

\subsection{Druckwasserreaktor}

\subsubsection{Allgemein}
\begin{itemize}[noitemsep]
	\item Kühlmittel: Wasser (entsalzt)
	\item Moderator: Wasser (entsalzt)
	\item Absorber: Silber, Indium, Cadmium
	\item Brennstoff: Angereichertes \(UO_2\)
	\item Wasser kocht nicht im Kern
	\item Dampf erst durch Dampferzeuger
	\item Druck: 16 MPa
	\item Temperatur: 300°C
	\item Thermische Leistung: 3.7 GW
	\item Effizenz: 35\%
\end{itemize}

\subsubsection{Sicherheit}
\begin{itemize}[noitemsep]
	\item Siehe BWR
	\item Unterschied Abschaltesystem
		\begin{itemize}[noitemsep]
			\item Steuerstäbe fallenlassen (werden durch Elektromagneten gehalten, Schwerkraft)
		\end{itemize}
\end{itemize}

\subsection{RBMK (Druckröhrenreaktor)}
\begin{itemize}[noitemsep]
	\item Kühlmittel: Wasser
	\item Moderator: Graphit (Brennbar)
	\item Absorber: Borcarbid
	\item Brennstoff: Angereichertes \(UO_2\)
	\item Kein Containment
	\item Langsames Schnellabschaltesystem (12-18s)
	\item Positiver Void-Koeffizent
\end{itemize}

\subsection{Diverse Reaktoren}

\subsubsection{CANDU (Natururanreaktor)}
\begin{itemize}[noitemsep]
	\item Kühlmittel: Schweres Wasser
	\item Moderator: Schweres Wasser
\end{itemize}

\subsubsection{Graphitmoderierter Reaktor (AGR)}
\begin{itemize}[noitemsep]
	\item Kühlmittel: \(CO_2\)
	\item Moderator: Graphit
	\item Höhere Temperaturen, besserer Dampf
\end{itemize}

\subsubsection{Hochtemperaturreaktor}
\begin{itemize}[noitemsep]
	\item Höhere Temperaturen (1000°C), für höhere thermische Effizenz
	\item \(CO_2\) nur bis 800°C, sonst desintegriert zu C und O
	\item Deswegen Helium: Stabil bei hohen T, chemisch inert, keine Aktivierung
	\item Keine metallische Brennstoffhülle möglich, nur dichtes Graphit
	\item Brennstoffkugeln, in Graphit gehüllt
\end{itemize}

\subsubsection{Schneller Brüter}
\begin{itemize}[noitemsep]
	\item Kühlmittel: Flüssiges Metall (Na, keine Moderation, heftige Reaktion mit Wasser)
	\item Moderator: Keiner
	\item Brutstoff: Natururan
	\item Brennstoff: MOX (Mixed Oxide Fuel), U-Pu-Mischoxid
	\item Kern innen: Spaltzone
	\item Kern aussen: Brutzone
\end{itemize}

\subsection{Reaktortypen - Verbreitung}
\begin{itemize}[noitemsep]
	\item PWR: \textasciitilde 270 (USA, Fr, Jp, Ru)
	\item BWR: \textasciitilde 90 (USA, Jp, Sw)
	\item CANDU: \textasciitilde 40 (Ca)
	\item GCR (Magnox, AGR): ~20 (UK)
\end{itemize}

\subsection{Anlagenzustände}
\begin{itemize}[noitemsep]
	\item Normalbetrieb, inkl. Störungen ohne Sicherheitsbedeutung
	\item Störfall
		\begin{itemize}[noitemsep]
			\item Betrieb kann Sicherheitsbedingt nicht fortgesetzt werden
			\item Anlage so konzipiert, dass Ereignis beherrscht werden kann
			\item Auslegungsstörfall (\textbf{DBA}, Design Basis Accident, früher GAU)
		\end{itemize}
	\item Unfall
		\begin{itemize}[noitemsep]
			\item Sehr kleine Eintrittswahrscheinlichkeit
			\item Keine gezielten Maßnahmen zur Verhinderung existent
			\item Auslegungsüberschreitender Störfall (\textbf{BDBA}, Beyond Design Basis Accident)
		\end{itemize}
	\item Schwerer Unfall: BDBA mit Kernschmelze
\end{itemize}

\subsection{Unfalleinteilungen}

\subsubsection{NRC (Nuclear Regulatory Commission)}
\begin{itemize}[noitemsep]
	\item Klasse 1: Trivialer Unfall
	\item Klasse 2: geringe Freisetzung außerhalb Containment
	\item Klasse 8: DBA: LOCA, Rod Drop Acc
	\item Klasse 9:
		\begin{itemize}[noitemsep]
			\item Gleichzeitige Failure On/Off-Site Power + LOCA
			\item Plötzliches RDB bersten
			\item Extrem unwahrscheinlich angenommen
		\end{itemize}
	\item Weiters: Spaltprodukte in Primär/Sekundärsystem, Refueling Acc, Spent Fuel Handling Acc
\end{itemize}

\subsubsection{IAEA}
\begin{itemize}[noitemsep]
	\item 0: keine, geringe Sicherheitstechnische Bedeutung
	\item 1-3 Störfall
		\begin{itemize}[noitemsep]
			\item Ausfall gestaffelter Sicherheitseinrichtungen
			\item Kontaminierung von Personal
			\item Sehr geringe Freisetzung
		\end{itemize}
	\item 4-6: Unfall
		\begin{itemize}[noitemsep]
			\item Schäden am Reaktorkern
			\item Tod von Personal
			\item Steigende Freisetzung in Umwelt
			\item Einsatz von Katastrophenmaßnahmen
		\end{itemize}
\end{itemize}

\section{Sicherheitrelevante Phänomene eines Reaktors}

\subsection{n-Haushalt}
\begin{itemize}[noitemsep]
	\item Abbrand: Brennstoff wird verbraucht
	\item Brüten: Brennstoff entsteht
	\item Vergiftung: Spaltprodukte mit großem Absorptionsquerschnitt
	\item Regelung: Absorberstäbe
\end{itemize}

\subsection{Unterscheide}
\begin{itemize}[noitemsep]
	\item Multiplikationsfaktor: k
	\item Reaktivität: \(\rho\)
	\item Reaktivitätskoeffizent: \(\alpha_x\)
\end{itemize}

\subsection{Reaktorkinetik}
\begin{itemize}[noitemsep]
	\item Multiplikationsfaktor k, gilt für alle n
	\item Prompte n: bei Kernspaltung erzeugt
	\item Verzögerte n: durch Zerfall von Spaltprodukten, \(\beta\)
	\item \textbf{Verzögert Überkritisch}
		\begin{itemize}[noitemsep]
			\item \(1 < k < 1 + \beta\)
			\item Reaktorleistung steigt an, aber nur durch verzögerte n
			\item Reaktor regelbar
		\end{itemize}
	\item \textbf{Prompt Kritisch}
		\begin{itemize}[noitemsep]
			\item \(k = 1 + \beta\)
			\item Grenze der Regelbarkeit
			\item Prompte n genügen zur Aufrechterhaltung der Kettenreaktion
			\item Unsicher, kleinste Änderung von k kann Anordnung Prompt Überkritisch machen
		\end{itemize}
	\item \textbf{Prompt Überkritisch}
		\begin{itemize}[noitemsep]
			\item \(k > 1 + \beta\)
			\item Exponentieller Leistungsanstieg
			\item Nicht Regelbar
		\end{itemize}
\end{itemize}

\subsection{Reaktivitätskoeffizent}
\subsubsection{Allgemein}
\begin{itemize}[noitemsep]
	\item \textbf{Reaktivität}: \(\rho = \frac{k_{eff} - 1}{k_{eff}}\) in \$/\%
	\item Abweichung des Multiplikationsfaktors \(k_{eff}\) von \(k_{eff} = 1\)
	\item \textbf{Reaktivitätskoeffizent}: \(\alpha_x = \frac{d\rho}{dX}\)
	\item Änderung der Reaktivität auf Veränderung von Prozessvariable X (T, p, Dichte, ...)
	\item Bsp: \(\alpha_{fuel}, \alpha_{clad}, \alpha_{mod}, \alpha_{void}\)
	\item \(\alpha_{tot} = \sum_{i}^{} \alpha_i\)
	\item \textbf{Ziel}: \(\alpha_{tot} < 0\) \textrightarrow Inhärent Sicher
\end{itemize}

\subsection{Xenonschwingungen}
\[
	Te^{135}
	\xrightarrow[< 2 min]{\beta^-}
	J^{135}
	\xrightarrow[\text{6.7 h}]{\beta^-}
	Xe^{135}
	\xrightarrow[\text{9.2 h}]{\beta^-}
	Cs^{135}
	\xrightarrow[2.6\cdot10^4a]{\beta^-}
	Ba^{135} \text{(Stabil)}
\]
\begin{itemize}[noitemsep]
	\item Fluß: Neutronenfluß/Neutronenflußdichte\\
		Summe aller Wege, die von n in einem Raum/Zeitbereich zurückgelegt werden
	\item Neutronengift: Xe-135, Sm-149, Pm-149, J-135
	\item Xe-135 entsteht aus 2-stufigem \(\beta^-\)-Verfall von Te-135 (0.003)
	\item Gleichgewichtigkeitskonzentration nach ca. 30 h erreicht
	\item Xenonaufbau Ortsabhängig
		\begin{itemize}[noitemsep]
			\item Zurerst in Flußhöherer Zone J-135
			\item Verzögert entsteht Xe-135
			\item Fluß verlagert sich in andere Zone \textrightarrow\ Schwingungen
		\end{itemize}
	\item Beim Abschalten (Fluß = 0)
		\begin{itemize}[noitemsep]
			\item J-135 zerfällt schneller (zu Xe-135) als Xe-135 (zu Cs-135) \textrightarrow\ Reaktivität sinkt
			\item Beim Hochfahren dann: Überschussreaktivität benötigt
		\end{itemize}
\end{itemize}

\subsection{Nuklearer Dopplereffekt}
\begin{itemize}[noitemsep]
	\item Je heißer Brennstoff, desto mehr Neutronen absorbiert
	\item Resultat: Weniger Reaktivität, passiert sofort
\end{itemize}

\subsection{Wigner-Energie}
\begin{itemize}[noitemsep]
	\item n-Energie umgewandelt in Deformationen des Kristallgitters (z.B. C)
	\item Wird über Wärmeausbrüche freigesetzt
	\item Gezielte Freisetzung (Ausglühen) möglich (\(\Delta T = 250K\))
\end{itemize}

\subsection{Sicherheitsfragen}
\begin{itemize}[noitemsep]
	\item Sicherstellung: System nie prompt Überkritisch
		\begin{itemize}[noitemsep]
			\item \(\alpha_{tot} < 0\)
			\item Sonst RIA \textrightarrow\ Reactivity Initiated Accident
		\end{itemize}
	\item Sicherstellung: keine Freisetzung von Spaltprodukten in Umwelt
		\begin{itemize}[noitemsep]
			\item 5 Barrieren
		\end{itemize}
\end{itemize}

\subsection{Sicherheitskonzepte}
\begin{itemize}[noitemsep]
	\item \textbf{Redundanz}
		\begin{itemize}[noitemsep]
			\item Wichtige Systemfunktion mehrfach vorhanden
			\item Sichere Schließfunktion: Serielle Ventile
			\item Sichere Öffnungsfunktion: Parallele Ventile
		\end{itemize}
	\item \textbf{Diversität}
		\begin{itemize}[noitemsep]
			\item Zwei Reaktorabschaltsysteme mit physikalisch unterschiedlicher Funktion
			\item Steuerstäbe
			\item Wasser mit Borzusatz
		\end{itemize}
	\item \textbf{Räumliche Trennung}
		\begin{itemize}[noitemsep]
			\item Sicherheitstechnische Komponenten räumlich getrennt
			\item Vierfach-Redundantes System, aufgeteilt in Vier Viertel
		\end{itemize}
	\item \textbf{Barrieren}
		\begin{itemize}[noitemsep]
			\item PWR/BWR: Brennstoffkristallgitter, Brennstabhülle, RDB, Sicherheitsbehälter
		\end{itemize}
	\item \textbf{Sicherheitsysteme zur Barrierenerhaltung (diversitär + redundant)}
		\begin{itemize}[noitemsep]
			\item \textbf{Kernnotkühlsysteme} (ECCS, Emergency Core Cooling Systems)
				\begin{itemize}[noitemsep]
					\item \textbf{Hochdruckeinspeisesystem (HPCI)},
						\begin{enumerate}[label = \textrightarrow]
							\item High Pressure Coolant Injection System
							\item wenn RDB unter Druck
							\item automatische Auslösung, wenn \(p_{kk} <11\) MPa
						\end{enumerate}

					\item \textbf{Niederdruckeinspeisesystem (LPCI)}
						\begin{enumerate}[label = \textrightarrow]
							\item Low Pressure Coolant Injection System
							\item wenn RDB ohne Druck
							\item automatische Auslösung, wenn \(p_{kk} = 0.9\) MPa
						\end{enumerate}

					\item \textbf{Core Spray System} (nur BWR)
						\begin{enumerate}[label = \textrightarrow]
							\item Sprüht auf Brennstäbe
							\item Verhinderung Dampfentstehung
						\end{enumerate}

					\item \textbf{Containment Spray System}
						\begin{enumerate}[label = \textrightarrow]
							\item Sprüht Kühlmittel in Sicherheitsbehälter
							\item Dampf soll kondensiert werden
							\item Verhinderung Überdruck/temperatur
						\end{enumerate}
				\end{itemize}

			\item \textbf{Nachwärmeabfuhrsysteme}
				\begin{itemize}[noitemsep]
					\item Essential Service Water System (ESWS)
						\begin{enumerate}[label = \textrightarrow]
							\item Wasser aus z.B. Fluss
							\item Rezirkulation über Kühltürme
						\end{enumerate}
				\end{itemize}

			\item \textbf{Emergency Electrical Systems}
				\begin{itemize}[noitemsep]
					\item Dieselgeneration
					\item Batterien
				\end{itemize}

		\end{itemize}
\end{itemize}

\subsection{Auslösende Ereignisse eines Unfalls}

\subsubsection{Transienten}
\begin{itemize}[noitemsep]
	\item Kurzeitige Änderung des Energielevels/eines Wertes
	\item z.B. weil Änderung Kühlmitteltemperatur/druck
\end{itemize}

\subsubsection{Interne Ereignisse}
\begin{itemize}[noitemsep]
	\item Station Blackout: Vollständiger Verlust d. Stromversorgung
	\item \textbf{ATWS}, Anticipated Transients withouth SCRAM
		\begin{itemize}[noitemsep]
			\item Unfall mit Behinderung der Wärmeabfuhr bei Versagen der Schnellabschaltung
		\end{itemize}
	\item \textbf{LOCA}, Kühlmittelverlust, Leck in
		\begin{itemize}[noitemsep]
			\item Hauptkühlmittelleitung,
			\item Druckhaltesystem (Defektes Ventil)
			\item Dampferzeuger
			\item RDB
			\item Anschlussleitung Kühlmittelkreislauf
		\end{itemize}
	\item \textbf{Transienten}, Änderung von
		\begin{itemize}[noitemsep]
			\item Leistungserzeugung
			\item Leistungsabfuhr (weil Änderung Speisewasserzufuhr/Dampfentnahme)
			\item Kühlmittelumwälzung
			\item Kühlmitteldruck
			\item Ausfall v. Sicherheitssystemen
		\end{itemize}
\end{itemize}

\subsubsection{Externe Ereignisse}
\begin{itemize}[noitemsep]
	\item Erdbeben, Tsunami, Tornado, Brand, Flugzeugabsturz
\end{itemize}

\subsection{Reaktorsicherheit - Forschungszentren}
\begin{itemize}[noitemsep]
	\item Forschungszentren Karlsruhe, heute KIT (D)
	\item Cadarache (F)
		\begin{itemize}[noitemsep]
			\item PHEBUS
		\end{itemize}
	\item Idaho National Engineering Laboratory (INEL, USA)
		\begin{itemize}[noitemsep]
			\item BORAX (BWR Experiments)
			\item PBF (Power Burst Facility)
			\item LOFT (Loss-of-Fluid Tests)
		\end{itemize}
\end{itemize}

\section{Sicherheitsstudien}

\subsection{Methoden d. Sicherheitsanalyse}
\begin{itemize}[noitemsep]
	\item Deterministisch: Auswirkung aufgrund v. Ursachen/Randbedingungen
	\item Probabilistisch: Technisches Versagen Frage der Zeit \textrightarrow\ Risikoberechnung
\end{itemize}

\subsection{Probabilistische Sicherheitsanalyse (PSA)}
\begin{itemize}[noitemsep]
	\item PSA-1: Ermittlung d. Häufigkeit von Kernschäden (CDF, Core Damage Frequency)
	\item PSA-2: Ermittlung d. Abläufe eines schweren Unfalls
	\item PSA-3: Analyse d. Folgen einer Ausbreitung von Radioaktivität in die Umwelt
\end{itemize}

\subsection{Ereignisbaumanalyse (ETA)}
\begin{itemize}[noitemsep]
	\item Methode zur Bestimmung möglicher Folgen eines Fehlers
	\item Binärer Baum, Unfallsequenzen
	\item Abzweigungen mit Fehlerwarscheinlichkeiten
	\item Multiplikation: Warscheinlichkeit für Pfad
\end{itemize}

\subsection{Fehlerbaumanalyse (FTA)}
\begin{itemize}[noitemsep]
	\item Methode zur Fehlerauffindung
	\item Liefert Kombinationsmöglichkeiten, die zu einem Fehler führen können
	\item Bool'sche Algebra
\end{itemize}

\subsection{Sicherheitsstudien}
\begin{itemize}[noitemsep]
	\item \textbf{WASH-740 (1957)}
		\begin{itemize}[noitemsep]
			\item Theoretical Possibilites \& Consequences of Major Accidents in large NPP
		\end{itemize}

	\item \textbf{WASH-1400 (1975)}
		\begin{itemize}[noitemsep]
			\item Reactor Safety Study
			\item Rasmussen Studie
		\end{itemize}

	\item \textbf{Dt. Risikostudie A (1979)}
		\begin{itemize}[noitemsep]
			\item Störfallbedingte Risiken durch KKW in D
			\item Ergebnisse v. WASH-1400 auf dt. KKW bezogen
		\end{itemize}

	\item \textbf{Dt. Risikostudie B (1989)}
		\begin{itemize}[noitemsep]
			\item Anlagentechnische Sicherheitsanalyse zur Ermittlung sicherheitsrelevanter Schwachstellen
		\end{itemize}

	\item \textbf{NUREG (ab 1981)}
		\begin{itemize}[noitemsep]
			\item Quelltermverhalten, immer wieder neu evaluiert
			\item NUREG/CR-4587 (1986): Source Term Code Package (STCP)
			\item NUREG-0956 (1986): Vorstellung SA-Codes
			\item NUREG-1150 (1987/90): Reactor Risk Reference Document (1987/90)
			\item NUREG-1465 (1995): Revised Source Terms
			\item NUREG-1935 (2009): SOARCA (State-of-the-Art Reactor Consequence Analysis)\\
				neue realistischere Betrachtung
		\end{itemize}

	\item Andere
		\begin{itemize}[noitemsep]
			\item OEFZS-A-2770: Internationale Quelltermstudien (1990)\\
				Erstmals für WWER, Wasser-Wasser-Energie-Reaktor, russischer PWR
			\item Erste Fr PSA: 1990, Updates 2003, 2005?
		\end{itemize}

\end{itemize}

\subsection{WASH-1400 (Rasmussen Studie)}

\subsubsection{Allgemein}
\begin{itemize}[noitemsep]
	\item 2 Referenzanlagen: Peach Bottom (BWR), Surry (PWR)
	\item 4 auslösende Ereignisklassen
		\begin{itemize}[noitemsep]
			\item LOCA
			\item Transienten
			\item RDB-Versagen
			\item Externe Ereignisse
		\end{itemize}
	\item 600 Unfallereignisse gerechnet
	\item Keine unwahrscheinliche Ereignisse
	\item Rechnungen mit Code CORRAL
	\item Quellterm eingeteilt
		\begin{itemize}[noitemsep]
			\item 9 Freigabekategorien für PWR
			\item 4 Freigabekategorien für BWR
		\end{itemize}
\end{itemize}

\subsubsection{Accident Sequence Symbols, Beispiele}
\begin{itemize}[noitemsep]
	\item PWR
		\begin{itemize}[noitemsep]
			\item A \textrightarrow\ Intermediate to large LOCA
			\item C \textrightarrow\ Failure of Containment Spray System
			\item TMLB'-\(\alpha\) \textrightarrow\ Station Blackout mit Containmentversagen durch Dampfexplosion
		\end{itemize}
	\item BWR
		\begin{itemize}[noitemsep]
			\item A \textrightarrow\ Rupture of reactor coolant boundary diameter \(>6\) inch
			\item E \textrightarrow\ Failure of Emergency Core Cooling Injection
		\end{itemize}
\end{itemize}

\subsubsection{Resultate}
\begin{itemize}[noitemsep]
	\item KS = Kernschmelze
	\item Hauptbeitrag zur KS-wahrscheinlichkeit: Transienten, kleine LOCAs
	\item Geschätzte Frequenz für KS höher als bisher angenommen
	\item Konsequenzen von KS geringer als bisher angenommen
	\item Risiko kleiner als in WASH-740 angegeben (Risiko = Eintrittwahrscheinlichkeit * Schwere)
	\item Bewertung 2011
		\begin{itemize}[noitemsep]
			\item Nicht mehr Aktuell
			\item Teilweise konservativ
			\item Andere Unfallabläufe durch Nachrüstungen
			\item Manche Abläufe nicht betrachtet
			\item Führte zu weltweiter Beschäftigung mit KKW-Risiko, Erhöhung d. Sicherheit
		\end{itemize}
\end{itemize}

\subsection{NUREG-1150 (Reactor Risk Reference Document)}
\begin{itemize}[noitemsep]
	\item Referenzanlagen: 5 typische US-KKWS, 3 PWR, 2 BWR
	\item 25 verschiedene Szenarien
	\item Rechnung mit STCP, MELCOR, CONTAIN
	\item Kritik: Annahmen zu Konservativ
\end{itemize}

\subsection{Deutsche Risikostudie Phase A}
\begin{itemize}[noitemsep]
	\item Referenzanlage: Biblis B
	\item Rechnung mit ALMOD (Anlagenverhalten), CORRAL II (Spaltproduktfreisetzung)
	\item 100 Sequenzen untersucht
	\item 8 Quellterm-Freisetzungskategorien
	\item Ergebnisse stimmen im wesentlich mit WASH-1400 überein\\
		aber andere Randbedingungen in D (Bauweisen, Bevölkerungsdichte)
	\item Großer Beitrag zur KS-Häufigkeit: kleine Lecks
\end{itemize}

\subsection{Deutsche Risikostudie Phase B}
\begin{itemize}[noitemsep]
	\item Referenzanlage: Biblis B
	\item Neubearbeitung unter Einbeziehung neuer Erkenntnisse
	\item 32 Auslösende Ereignisse
	\item 19 Schadenszustände bei Ausfall Wärmeabfuhr
	\item 6 Kernschmelzfälle, z.B.
		\begin{itemize}[noitemsep]
			\item ND: Niedriger Druck in Primärkreis, nach früher Entlastung durch auslösendes Ereignis
			\item ND*: Niedriger Druck in Primärkreis, nach später Entlastung durch Notfallmaßnahmen
			\item HD: Hoher Druck in Primärkreis
			\item PLR-ND/ND*: Leck Primärkreis Ex-Containment
			\item DE-ND*/HD: Leck Dampferzeuger
		\end{itemize}
	\item Freisetzungsmöglichkeiten, z.B.
		\begin{itemize}[noitemsep]
			\item Allgemein: Containment: Versagen, Umgehung, Gezielte Freisetzung
			\item AF-SBV: Großflächiges Sicherheitsbehälterversagen oberhalb der Fundamentplatte
			\item AF-DF: Freisetzung ins Grundwasser, infolge durchschmelzens des SB und Fundamentplatte
		\end{itemize}
	\item Dominant: kleine Lecks
	\item Unterschiede zu Phase A: Weniger Freisetzung, wegen Systemverbesserungen
\end{itemize}

\section{Brennstoffverhalten und Kernversagen}

\subsection{Barriere 1 - Brennstoff}

\subsubsection{Brennstoffarten}
\begin{itemize}[noitemsep]
	\item Pellets, gestapelt in Brennstab
	\item Oxide: \(UO_x, MO_x\)
	\item Metall: Aktinide
	\item Keramik: Urankarbid
	\item Flüssig: Uranlysalz
\end{itemize}

\subsubsection{Urandioxid}
\begin{itemize}[noitemsep]
	\item Schmelzpunkt: ~3200K
	\item Pellets bei 95\% theoretischer Dichte
	\item Freies Volumen für Spaltproduktfreisetzung
	\item Konkave Oberfläche, Ausdehnung oben/unten
\end{itemize}

\subsubsection{Spaltproduktinventar}
\begin{itemize}[noitemsep]
	\item 100\% Freisetzung: Edelgase (Xe, Kr)
	\item Hoch Flüchtig: Halogene, Alkalimetalle (I, Br, Cs)
	\item Flüchtig: Erdalkali, Metalllegierungen
	\item Wenig Flüchtig: Oxide (Zr)
\end{itemize}

\subsubsection{Freisetzung in intaktem Brennstoff}
\begin{itemize}[noitemsep]
	\item Diffusion, Knockout, Recoil
	\item Swelling: Gasblasen in Pellet
	\item Cracking (Spaltbildung) v. Pellets bei \(T < 1000\)°C (keine Plastizität)
	\item Kriechen (Verformung) v. Pellets bei \(T > 1100\)°C (Plastizität)
	\item Spalt zw. Pellet-Cladding (He Gas) \textrightarrow\ thermischer Widerstand
	\item Bei Ausdehnung: Bildung v. Cracks \textrightarrow\ Verlagerung Pelletsegmente
		\begin{itemize}[noitemsep]
			\item Blockierung des Hüllrohrs
			\item Anstieg Wärmefluss im Spalt
			\item Verringerung Wärmefluss bei \(UO_2\) weil Spalt
			\item Bildung von Xe/Kr im Spalt \textrightarrow\ Änderung Wärmeleitfähigkeit
		\end{itemize}
\end{itemize}

\subsection{Barriere 2 - Hüllrohr}

\subsubsection{Allgemein}
\begin{itemize}[noitemsep]
	\item Material: Zircaloy (-2 BWR, -4 PWR)
	\item Phasenumwandlung: ~1200 K
	\item Schmelzpunkt: ~2100 K
	\item Testanlagen: NIELS, REBEKA, PBF, PHEBUS
\end{itemize}

\subsubsection{Physikalische Effekte}
\begin{itemize}[noitemsep]
	\item Thermische Ausdehnung (Umfang)
	\item Verlagerung der Pellets (assymetrische T-Verteilung \textrightarrow\ Spannungen)
	\item Druck/Spannung durch Kühlmittel von Aussen
	\item Kriechen durch n-Fluss
	\item Kalte Regelstäbe
		\begin{itemize}[noitemsep]
			\item Tests: REBEKA
			\item Brennstäbe um kalten Regelstab
			\item Bersten auf Außenseite
		\end{itemize}
	\item Deformation \textrightarrow\ Blockierung von Kühlkanälen
\end{itemize}

\subsubsection{Effekte im Spalt}
\begin{itemize}[noitemsep]
	\item Spaltproduktfreisetzung von Pellet in Spalt \textrightarrow\ Druckanstieg \textrightarrow\ \textbf{Ballooning}
	\item Pelletausdehnung \textrightarrow\ Kontakt \textrightarrow\ Spannungen \textrightarrow\ \textbf{Bambooing}
	\item Bei kritischen Spannungen: Riss im Hüllrohr
\end{itemize}

\subsubsection{Chemische Effekte}
\begin{itemize}[noitemsep]
	\item Hüllrohr-Oxidation
		\begin{itemize}[noitemsep]
			\item Zircaloy-Dampf-Reaktion
			\item stark T abhängig, exotherm
			\item ab ~1500°C zusätzlicher Anstieg der Oxidationsrate
		\end{itemize}
	\item \(UO_2-Zry\) Wechselwirkung
		\begin{itemize}[noitemsep]
			\item Entstehung von \(ZrO_2\)
		\end{itemize}
	\item \(UO_2\) Auflösung durch geschmolzenes Zircaloy
		\begin{itemize}[noitemsep]
			\item Tests: CORA-13
			\item Abhängig von Oxidation
			\item Beim Schmelzen v. Zry: \(UO_2, ZrO_2\) weit unter Schmelzpunkt verflüssigt
			\item Bei tieferen T: Verfestigung, Blockierung v. Kühlkanälen
		\end{itemize}
	\item \(H_2\) Produktion
	\item Air-Ingress
\end{itemize}

\subsubsection{Temperaturverhalten}
\begin{itemize}[noitemsep]
	\item 1100K: Schmelpunkt Ag, In, Cd
	\item ~1200K:
		\begin{itemize}[noitemsep]
			\item Bersten d. Brennstäbe, Beginn Spaltproduktfreisetzung
			\item Verstärkte Zry-Oxidation
		\end{itemize}
	\item ~1400K:
		\begin{itemize}[noitemsep]
			\item Zerstörung BWR Steuerstäbe, Stahl-Borkarbid Eutektikum
			\item Zerstörung PWR Steuerstäbe, Zry-Ag Eutektikum
			\item Zry-Dampf-Reaktion
			\item Absorbermaterial schmilzt
		\end{itemize}
	\item \textasciitilde 1700K: Beginn Auflösung Pellet \& Hüllrohr, Schmelzpunkt, Edelstahl
	\item \textasciitilde 1800K: Eskalation Zry-Oxidation
	\item \textasciitilde 2000K: Schmelzpunkt Zry
	\item 2100K: Verflüssigung \(UO_2\)-Zry
	\item 2400-2600K: Zerstörung Brennstäbe
	\item \textasciitilde 3000K: Schmelzpunkt \(ZrO_2\)
	\item \textasciitilde 3100K: Schmelzpunkt \(UO_2\)
\end{itemize}

\subsubsection{Siedearten}
\begin{itemize}[noitemsep]
	\item Naturkonvektion
	\item Keimsieden:
		\begin{itemize}[noitemsep]
			\item DNB \textrightarrow Departure of Nucleate Boiling
			\item Dampfblasen an Oberfläche \textrightarrow Isolierend
			\item Wärmeübertragung sinkt
		\end{itemize}
	\item Übergangssieden
		\begin{itemize}[noitemsep]
			\item Dampffilm und Flüssigkeitsschicht
			\item Wärmeübertragung sinkt
			\item Minimum bei Leidenfrost-Punkt
		\end{itemize}
	\item Stabiles Filmsieden
		\begin{itemize}[noitemsep]
			\item Stabiler Dampffilm
			\item Wärme nur durch Strahlung übertragen
			\item Heizer brennt durch
		\end{itemize}
	\item Ziel: Wärmeübertragung \(<\) DNB
\end{itemize}

\subsubsection{Strömungformen beim Strömungssieden}
\begin{itemize}[noitemsep]
	\item Oberflächensieden: Blasen kondensieren wieder
	\item Pfropfenbildung der Blasen
	\item Ringströmung (Filmsieden): Zentrum d. Kanals voller Dampf, Flüssigkeitsfilm an Wand
	\item Austrocknungspunkt: Ende d. Flüssigkeitsfilms
	\item Nebelstrom: nur Tröpfchen
	\item Dampzustand
	\item Haben verschiedenste Gesetze zur Wärmeübertragung
\end{itemize}

\subsubsection{Kühlmechanismem beim Fluten (Quenchen)}
\begin{itemize}[noitemsep]
	\item Wasser
	\item Wasser bei Keimsieden
	\item Wasser bei Filmsieden
	\item Wassertropfen + Wasserdampf
	\item Wasserdampf
\end{itemize}

\subsubsection{\(H_2\) Produktion}
\begin{itemize}[noitemsep]
	\item \(Zr + 2 H_2O \Leftrightarrow ZrO_2 + 2 H_2 + \Delta\)
	\item Exotherme Reaktion, bei hohen T
	\item Bei LOCA, Fluten
\end{itemize}

\subsubsection{Air-Ingress}
\begin{itemize}[noitemsep]
	\item Eindringen von Luft in System
	\item Oxidation viel besser als mit Dampf
	\item Reaktionswärme 85\% höher \textrightarrow\ beschleunigt Oxidation
	\item \(O_2\) Arme Atmosphäre: Zry \textrightarrow\ Zirkonnitrid\\
		\textrightarrow\ hoch entflammbar, explosiv bei \(0_2\)/Dampfzufuhr
	\item \(O_2\) erzeugt bei \(UO_2\) verstärkt Spaltgase (Ru)
	\item Verschiedene Dichte v. Zr, \(ZrO_2\), ZrN\\
		\textrightarrow\ Stress \textrightarrow\ Risse \textrightarrow\ Hüllrohrversagen
\end{itemize}

\subsection{Barriere 3 - Druckbehälter}

\subsubsection{Allgemein}
\begin{itemize}[noitemsep]
	\item Material: ferritscher Stahl, innen Cr/Ni Anteil
	\item Dicke: 20-30cm
	\item Primärkreisläufe: meist 4
	\item Versprödet durch n
	\item Heilt durch \(\Delta T\)
	\item BWR größer als PWR
	\item Testanlagen: LOFT, CORA
\end{itemize}

\subsubsection{Physikalische Effekte}
\begin{itemize}[noitemsep]
	\item Schmelze ab \textasciitilde 2700K
	\item Bewegt sich nach unten
	\item Strukturmaterial kann Wärme aufnehmen \textrightarrow Verfestigung möglich
	\item Schmelzsee sammelt sich unten, Gefährdet Integrität d. Behälters
		\begin{itemize}[noitemsep]
			\item DCH (Direct Containment Heating)
			\item Dampfexplosion
			\item Wechselwirkung mit Corium
		\end{itemize}
	\item Kern tropft herunter (Candling)
\end{itemize}

\subsubsection{Physikalische Effekte bei Maßnahmen}
\begin{itemize}[noitemsep]
	\item Muss bei Erstellung von Unfallrichtlinien beachtet werden
	\item Wasser ja/nein?
	\item Bei Kühlung durch Wasser (Fluten)
		\begin{itemize}[noitemsep]
			\item Gefahr T-Spitze
			\item \(H_2\) Spitze + p-Spitze
			\item Knallgasexplosion
			\item Dampfexplosion
		\end{itemize}
\end{itemize}

\subsubsection{Wärmeübergänge}
\begin{itemize}[noitemsep]
	\item Strahlung
	\item Konvektion (nach oben)
	\item Konvektion zu Behälter
	\item bei \(\Delta T\) Ausdehnung Behälter, Spaltentstehung (asymetrische T Verteilung)
\end{itemize}

\subsubsection{Bruch}
\begin{itemize}[noitemsep]
	\item Duktiles/Zähes Versagen
		\begin{itemize}[noitemsep]
			\item Elastische Verformung
			\item Bis erreichen d. Bruchspannung
		\end{itemize}
	\item Sprödes Versagen
		\begin{itemize}[noitemsep]
			\item Entstehung/Wachsen von Rissen
		\end{itemize}
	\item Bei mehr T
		\begin{itemize}[noitemsep]
			\item Ab 800K \textrightarrow\ Verlust Festigkeit d. Stahlbehälters
			\item Versagen d. Kriechen (Verformung unter Last)
		\end{itemize}
\end{itemize}

\subsubsection{Bruch - Schmelzfreisetzung}
\begin{itemize}[noitemsep]
	\item Wenn \(p_{RDB} \text{\textasciitilde} p_{Cont}\)\\
		\textrightarrow\ Herunterrinnen, Schwerkraft, geringe Aerosolfreisetzung
	\item Wenn \(p_{RDB} >> p_{Cont}\ (10MPa)\)\\
		\textrightarrow\ Austoss d. Schmelze, Menge abhängig von Lage d. Lecks
\end{itemize}

\subsection{Barriere 4 - Sicherheitsbehälter}

\subsubsection{Allgemein}
\begin{itemize}[noitemsep]
	\item Tests: SNL (F4 (Flugzeug) \textrightarrow\ Behälter)
	\item Testprobleme: Skalierung, Übertragung auf andere Größenordnungen
\end{itemize}

\subsubsection{Einrichtungen Reduktion p/T}
\begin{itemize}[noitemsep]
	\item Spray-System
	\item Fan-Cooler System
	\item Ice Condenser \textrightarrow Absorbiert Dampf durch Eisblöcke
	\item Wärmekapazität d. Wände
	\item Wetwell: Wassergefüllter Torus, Dampf wird dort kondensiert
	\item Ausgelegt auf Druck bei Bruch d. Hauptkühlmittelleitung
\end{itemize}

\subsubsection{Typen}
\begin{itemize}[noitemsep]
	\item PWR - Large Dry Containment
	\item PWR - Subatmospheric Containment
	\item PWR - Ice Condenser Containment
	\item BWR - Mark I/II/III
		\begin{itemize}[noitemsep]
			\item Bei Mark I
			\item Drywell (unter RDB) und Wetwell
		\end{itemize}
\end{itemize}

\subsubsection{Versagensarten}
\begin{itemize}[noitemsep]
	\item Bypass
		\begin{itemize}[noitemsep]
			\item Versagen d. Isolation des Containments
			\item SGTR (Steam Generator Tube Rupture)
			\item Externe Ereignisse (Flugzeugabsturz, Erdbeben)
		\end{itemize}
	\item Frühes Versagen
		\begin{itemize}[noitemsep]
			\item Überdruck (Gase, \(H_2\)/Dampf-Explosion, DCH \textrightarrow Direct Containment Heating)
			\item Geschosse durch Dampfexplosion
			\item Wechselwirkung m. Schmelze
			\item Stoß d. Druckbehälters
		\end{itemize}
	\item Spätes Versagen
		\begin{itemize}[noitemsep]
			\item Überdruck (Gas/Dampf, \(H_2\)-Explosion)
			\item Durchschmelzen \textrightarrow\ MCCI
			\item Strukturversagen: Erosion
		\end{itemize}
\end{itemize}

\subsubsection{Direct Containment Heating (DCH)}
\begin{itemize}[noitemsep]
	\item Austoss d. Schmelze + Dampf aus RDB
	\item High Pressure Melt Ejection (HPME)
	\item Schmelze zerstäubt
	\item Aufheizung d. Behälters
\end{itemize}

\subsection{Barriere 5 - Reaktorgebäude}

\subsubsection{Allgemein}
\begin{itemize}[noitemsep]
	\item Unterdruckzone
\end{itemize}

\subsubsection{Versagensarten}
\begin{itemize}[noitemsep]
	\item \(H_2\) Explosion
	\item Öffnungen, Undichtheit
	\item Externe Ereignisse (Flugzeugabsturz, Erdbeben, Geschosse, Tornado)
\end{itemize}

\section{Aerosolverhalten}

\subsection{Allgemein}
\begin{itemize}[noitemsep]
	\item Aerosol: In der Luft vorhandene Partikel
	\item Koagulation: Verklumpung kleiner Teilchen
	\item Parameter: Teilchengröße, Oberflächenbeschaffenheit, Chem. Zusammensetzung
\end{itemize}

\subsection{Quellen}
\begin{itemize}[noitemsep]
	\item Mechanische Erzeugung bei SA vernachlässigt
	\item Hauptquelle bei SA: Keimbildung von übersättigtem Dampf
	\item Materialquellen
		\begin{itemize}[noitemsep]
			\item Haupsächlich: Beton
			\item Brennstoff + Strukturelemente
			\item Fissionsprodukte
		\end{itemize}
	\item Dämpfe bei hohen T aus Material enstanden
	\item Transport in kältere Regionen \textrightarrow\ Übersättigung \textrightarrow\ Aerosolbildung
\end{itemize}

\subsection{Resuspension}
\begin{itemize}[noitemsep]
	\item Abgelagertes Aerosol wieder aufgewirbelt
	\item Durch Anstieg d. Gasflusses, Schock/Vibration d. Oberfläche
	\item Ursachen
		\begin{itemize}[noitemsep]
			\item Dampferzeugung im Core
			\item Bruch d. Druckbehälters
			\item Fuel-Coolant Wechselwirkung \textrightarrow\ Dampfexplosion
			\item \(H_2\)-Explosion
		\end{itemize}
\end{itemize}

\subsection{Senkung d. Aerosolmenge}
\begin{itemize}[noitemsep]
	\item Containment-Sprays
	\item Fan Coolers
	\item Steam Suppression Pools
\end{itemize}

\subsection{Phänomene im RDB}
\begin{itemize}[noitemsep]
	\item Keimbildung v. Dämpfen
	\item Kondensation v. Dämpfen an Aerosolen/Oberflächen
	\item Agglomeration v. Aerosolen
	\item Deposition (Ablagerung) v. Aerosolen
\end{itemize}

\subsection{Aersole bei MCCI}
\begin{itemize}[noitemsep]
	\item Abhängig v. Betonart
		\begin{itemize}[noitemsep]
			\item Kalkhaltig (BWR, PWR)
			\item Siliziumhaltig (PWR)
			\item Sehr viel Aerosole bei BWR Kalkhaltig
		\end{itemize}
	\item Freigesetzte Komponenten
		\begin{itemize}[noitemsep]
			\item Freies \(H_2O\): bei 230°C
			\item Gebundenes \(H_2O\): 300-450°C
			\item \(CO_2\): 600-1100°C
		\end{itemize}
	\item Testanlagen: PHEBUS
\end{itemize}

\section{Iodverhalten}

\subsection{Allgemein}
\begin{itemize}[noitemsep]
	\item Radiologisch signifikant
	\item Schilddrüse benötig Iod \textrightarrow\ Prophylaxe Kaliumiodid
	\item Bei \(1300 MW_{el}\) PWR: 15kg Iod
	\item Hohe Freisetzungsrate
	\item Hohe Flüchtigkeit
	\item Komplexe Transportmechanismen unter Strahlung, Ag in Kontrollstäben
	\item Forschung: sehr konservative Annahmen teilweise
	\item Forschung ab TMI (1979) intensiviert
	\item PHEBUS: State-of-the-Art-Report
	\item Modelle: COCOSYS, ASTEC
\end{itemize}

\subsection{Vorraussetzungen}
\begin{itemize}[noitemsep]
	\item Freisetzung v. Spaltprodukten (Edelgase, I, Cs)
	\item Freisetzung nicht radioaktiver Komponenten (Ag, In, Metalle)
	\item Transport in Primärkreis
		\begin{itemize}[noitemsep]
			\item \(H_2O/H_2\) als Dampf/Aerosol
			\item AgI-Dampf (hohe T)
			\item CsI-Aerosol
		\end{itemize}
\end{itemize}

\subsection{Flüchtigkeitsfaktoren}
\begin{itemize}[noitemsep]
	\item Oberflächenmaterialien/Farben
		\begin{itemize}[noitemsep]
			\item Haupteinfluss: Änderung Sumpfwasser ph-Wert
			\item Vor allem organische Oberflächen beeinflussen pH-Wert \textrightarrow Wasserunreinheiten
			\item Manche Oberflächen absorbieren viel Iodid
			\item Entstehung verschiedener Iod-Ausprägunsformen (Gas/Wasser/Oberfläche)
		\end{itemize}
	\item pH-Wert
		\begin{itemize}[noitemsep]
			\item \textbf{Niedriger pH-Wert\textrightarrow Mehr Iodid-Flüchtigkeit}
			\item Mehr pH \textrightarrow Mehr \(I^-\) verbleibt in Wasser
		\end{itemize}
	\item Präsenz von Strahlung
		\begin{itemize}[noitemsep]
			\item \textbf{Strahlung erhöht Iodid-Flüchtigkeit stark}
			\item Mehr Adsorption in Gasphase (statt Flüssigphase)
		\end{itemize}
	\item \textbf{Temperatur beeinflusst Iodid-Flüchtigkeit kaum}
\end{itemize}

\subsection{Iod-Chemie}
\begin{itemize}[noitemsep]
	\item Von Brennstoffzellen als \(CsI\) in Containment \textrightarrow Wasserlöslich
	\item Enstehung von nicht-flüchtigem \(I^-\) \textrightarrow Radiolyse \textrightarrow Entstehung flüchtiger Iodide
	\item Flüssigphase
		\begin{itemize}[noitemsep]
			\item Radiolyse v. Wasser in \(OH, H, H_2, HO_2, H_2O_2\)
			\item Flüchtig: \(I_2, RI\)
			\item Nicht-Flüchtig: \(I^-, I^-_3, IO^-_x, HOI\)
			\item \(I_2\) einfach von Wasser in Gasphase
			\item Einige \(RI\) sehr flüchtig \textrightarrow Gasphase
		\end{itemize}
	\item Gasphase
		\begin{itemize}[noitemsep]
			\item Radiolyse v. Trockener Luft: \(N_2, O_2, H_2O \rightarrow OH, HO_2, O, O_3, N\)
		\end{itemize}
\end{itemize}

\subsection{Iod-Thermodynamik}
\begin{itemize}[noitemsep]
	\item Sinkende T in Primärkreis: Bildung I-transportiernder Teilchen
		\begin{itemize}[noitemsep]
			\item Reaktion Metall-Iodid Dämpfe mit Aerosolen
			\item Reaktion m. Spaltprodukten
			\item Kondensation an Aerosoloberflächen
		\end{itemize}
	\item 60-70\% d. Iods erreicht Containment als Aerosol, 2-4\% als Dampf (\(HI, CdI_2\))
	\item Ablagerung durch Gravitation, Diffusiophorese (Kondensation an kälterer Wand)
\end{itemize}

\subsection{Iod-Silber Reaktionen}
\begin{itemize}[noitemsep]
	\item Reaktion mit \(Ag/AgO_2\) \textrightarrow\ I-Senke
	\item AgI nicht flüchtig, gering Löslich in Wasser
	\item Wirksamkeit v. Menge/Art Ag abhängig
\end{itemize}

\subsection{Entfernung durch Spraysysteme}
\begin{itemize}[noitemsep]
	\item Tropfen Absorptionsfähigkeit/Größe
	\item Chemische Zusätze unterstützend
	\item Hoher pH-Wert \textrightarrow\ unterstützt Hydrolyse
\end{itemize}

\section{Wasserstoffverhalten}

\subsection{Allgemein}
\begin{itemize}[noitemsep]
	\item Testanlagen: CONAN, REKO, MISTRA
	\item Modelle/Codes: COCOSYS, GASFLOW
\end{itemize}

\subsection{Quellen}
\begin{itemize}[noitemsep]
	\item In-Vessel
		\begin{itemize}[noitemsep]
			\item Zircaloy-Oxidation (1000MW PWR \textrightarrow\ 1T \(H_2\), BWR doppelt)
			\item Stahl-Oxidation
			\item Borkarbid-Oxidation
			\item Oxidation beim Fluten
			\item Kernschmelze
			\item Fuel-Coolant Wechselwirkung (Schmelze in Wasser d. unteren Plenums)
		\end{itemize}
	\item Ex-Vessel
		\begin{itemize}[noitemsep]
			\item Radiolyse v. Wasser
			\item Korrosionsreaktionen (Zn, Al)
			\item Reaktion \(UO_2\) mit Dampf, Wasser
			\item MCCI (Zr/Si/Cr/Fe + \(H_2O/CO_2\) oxidieren \textrightarrow\ \(H_2\))
			\item Trümmer/Atmosphäre Reaktion (HPME + Atmosphäre \textrightarrow\ \(H_2\))
		\end{itemize}
\end{itemize}

\subsection{Maßnahmen}
\begin{itemize}[noitemsep]
	\item Inertialisierung d. Containments (\(N_2, CO_2\))
	\item Katalytische Rekombination
	\item Zünden
\end{itemize}

\section{Dampfexplosion}

\subsection{Allgemein}
\begin{itemize}[noitemsep]
	\item Plötzlicher Übergang Flüssigkeit \textrightarrow\ Dampf
	\item Fuel-Coolant Interaction (FCI)
	\item Testanlagen: KROTOS, TROI
	\item Modelle/Codes: IKEJET, MATTINA
	\item Bei Tests: Skalierungsprobleme
\end{itemize}

\subsection{Ablauf}
\begin{itemize}[noitemsep]
	\item Schmelze fließt in Wasser
	\item Umwandlung in Tröpfchen
	\item Kontakt mit Wasser: Filmsieden
	\item Stabile Situation, weil langsamer Wärmeübergang
	\item Trigger erzeugt Schockwelle \textrightarrow\ Filmsieden destabilisiert \textrightarrow\ Weitere Fragmentierung/Druckaufbau
	\item Explosionswelle erzeugt weitere Fragmentierung \textrightarrow\ Mehr Oberfläche \textrightarrow\ Besserer Wärmeübergang
	\item Explosion
	\item Pre-Mixing-Phase: 0.1-10s
	\item Explosionsphase: Sekundenbruchteile
\end{itemize}

\subsection{Parameter}
\begin{itemize}[noitemsep]
	\item Wasser: Tiefe, m, T
	\item Schmelze: m, T, Geschwindigkeit, Druck
	\item Behälter: Geometrie
	\item Trigger: Timing, Stärke
	\item Experimente: Explosion nur ab Grenztemperatur möglich
\end{itemize}

\subsection{Szenarien}
\begin{itemize}[noitemsep]
	\item Keine Dampfexplosion, verstärktes Sieden
	\item Eine/Wenige schwache Dampfexplosionen
	\item Eine große Dampfexplosion
\end{itemize}

\subsection{Folgen}
\begin{itemize}[noitemsep]
	\item Bruch RDB \textrightarrow\ HPME, Schmelzaustritt durch Gravitation
	\item Teile d. RDB als Geschosse \textrightarrow\ Containmentversagen
	\item Rocket-Case: MC als Geschoss oben aus RDB
\end{itemize}

\subsection{Möglichkeiten}
\begin{itemize}[noitemsep]
	\item Kein experimenteller Nachweis v. Dampfexplosion bei \(UO_2\)/Corium mit Wasser\\
		\textrightarrow\ Sehr schnelle Krustenbildung
	\item Nachweise bei \(Al_2O_3\) in Wasser\\
		\textrightarrow\ langsamere Krustenbildung
\end{itemize}

\section{Molten Core Concrete Interaction (MCCI)}

\subsection{Beton}
\begin{itemize}[noitemsep]
	\item Zement als Bindemittel
	\item Zuschlag, verschiedene Größen
	\item Mit Wasser \textrightarrow\ Matrix mit Zuschlagsmaterial gefüllt
	\item Calciumsilicathydrat (CSH)
	\item Verschiedene Arten
		\begin{itemize}[noitemsep]
			\item Schwerer Beton: Verdampfung Kristallwasser: 100°C
			\item Superschwerer Beton
			\item Serpentinistischer Beton: Verdampfung Kristallwasser: 500°C, n-Moderation
		\end{itemize}
\end{itemize}

\subsection{Beton und Wasser}
\begin{itemize}[noitemsep]
	\item Freies Wasser: in Poren des Zementsteins
	\item Physikalisch gebundenes Wasser: an freien Oberflächen abgelagert
	\item Chemisch gebundenes Wasser: Hydrationsprodukte d. Zements
\end{itemize}

\subsection{Phänomene}
\begin{itemize}[noitemsep]
	\item Korrosion
	\item Freisetzung v. Dampf/Gasen aus sich zersetzendem Beton
	\item Chemische Reaktion der verdampften Bestandteile mit Schmelze
	\item Vermischung d. Kernschmelze mit Betonschmelze \textrightarrow\ Änderung Erstarrungsverhalten
	\item Wet Cavity
		\begin{itemize}[noitemsep]
			\item Dampfspitze \textrightarrow\ Containmentbruch
			\item Dampfexplosion
			\item Geschossbildung aus Trümmern
			\item Wasserstoffexplosion
		\end{itemize}
\end{itemize}

\subsection{Sicherheitsfragen}
\begin{itemize}[noitemsep]
	\item Größe d. freigesetzten Spaltprodukte
	\item Überdruckversagen d. Containments möglich?
	\item Durchschmelzen d. Betonwanne?
	\item Einbruch Gebäudestrukturen innerhalb Containments?
\end{itemize}

\subsection{Corium in Cavity}
\begin{itemize}[noitemsep]
	\item Corium bei 2700K
	\item Metallische Phase: Stahl + Zircaloy
	\item Oxidationsphase: \(UO_2 + ZrO_2\)
	\item Energie
		\begin{itemize}[noitemsep]
			\item PWR \(1000MW_{el}\): \(250\frac{W}{kg}\) (early), \(50\frac{W}{kg}\) (12 days), \(5\frac{W}{kg}\) (1 year)
			\item BWR \(1000MW_{el}\): \(160\frac{W}{kg}\) (early), \(30\frac{W}{kg}\) (12 days), \(3\frac{W}{kg}\) (1 year)
		\end{itemize}
	\item Modellierung
		\begin{itemize}[noitemsep]
			\item Homogene Konfiguration: Oxidschicht
			\item Geschichtete Konfiguration: Kruste \textrightarrow\ Oxidschicht \textrightarrow\ Metallschicht
		\end{itemize}
\end{itemize}

\subsection{Interaktionen}
\begin{itemize}[noitemsep]
	\item Schichten bekommen Masse durch zersetzenden Beton (\(H_2O,\ CO_2\), Schlacke)
	\item Oxidschicht
		\begin{itemize}[noitemsep]
			\item Strahlung(Wärme) \textrightarrow\ Atmosphäre, Metallschicht, Beton
			\item \(H_2,\ CO,\ CO_2,\ H_2O\) \textrightarrow\ Atmosphäre
			\item Reduzierte Oxide \textrightarrow\ Metallschicht
		\end{itemize}
	\item Metallschicht
		\begin{itemize}[noitemsep]
			\item Wärme \textrightarrow\ Beton, Oxidschicht
			\item Oxidierte Metalle, Gas \textrightarrow\ Oxidschicht
		\end{itemize}
\end{itemize}

\subsection{Chemie d. Schmelze}
\begin{itemize}[noitemsep]
	\item Oxidation:
		\begin{itemize}[noitemsep]
			\item (Zr/Cr/Fe) + (\(H_2O/CO_2\)) \textrightarrow\ \(H_2/CO\)
		\end{itemize}
	\item Reduktion:
		\begin{itemize}[noitemsep]
			\item \(Zr + (SiO_2/Fe_2O_3) \rightarrow (ZrO_2/Fe/Si/SiO)\)
			\item \(Si + (H_2O/CO_2) \rightarrow (H_2/CO)\)
		\end{itemize}
	\item Resultat: \textbf{Wasserstoffproduktion}
\end{itemize}

\subsection{Tests/Modelle}
\begin{itemize}[noitemsep]
	\item Testanlagen: COMET, VULCANO, BETA I
	\item Modelle/Codes: CORCON, ASTEC-MEDICIS
	\item COMET-L2
		\begin{itemize}[noitemsep]
			\item Durch Induktionswärme
			\item Keine Oxidschmelze zu Beginn
			\item Geschmolzener Beton sammelt sich auf Metallschmelze
			\item Anfangs Radial/Axial-Erosion 1:1
			\item Nach 100s: Axiale Erosion 2/3 x schneller
			\item Über längere Zeit: Perioden moderater Interaktion unterbrochen durch Schmelzeruptionen
				\textrightarrow\ Brechen v. Metallkrusten
		\end{itemize}
\end{itemize}

\section{Quelltermverhalten}

\subsection{Simulation d. Computercodes}
\begin{itemize}[noitemsep]
	\item Integrale Codes: STCP, ESCADRE, MELCOR, ASTEC (vs. Single Effect Codes)
	\item Grundlagen sind Experimente (PBF, LOFT, PHEBUS)
	\item Vorläufer: CORRAL (1977), MARCH (1980), MARCH-2 (1984)
\end{itemize}

\subsection{STCP}

\subsubsection{Allgemein}
\begin{itemize}[noitemsep]
	\item Komponenten für
		\begin{itemize}[noitemsep]
			\item Primärsystem/Coreverhalten
			\item Ex-Vessel Aerosole
			\item Transport Radionuklide
			\item Containment
		\end{itemize}
	\item Modellierung: Nodalisierung
\end{itemize}

\subsubsection{Ergebnis Primärkreislauf}
\begin{itemize}[noitemsep]
	\item Sequenz: AB (Large Break + Station Blackout)
	\item Szenario (Primärkreis)
		\begin{enumerate}
			\item Bruch
			\item 5.8MPa: Einspeisung Akkumulatoren
			\item Kaltes Wasser erhitzt (Druckanstieg)
			\item 800s: Kernfreilegung (reduzierung Wärmetransfer/Dampferzeugung)
			\item 3300s: Kern fällt herunter (Dampferzeugung)
			\item 4000s: Gleichgewicht Energieerzeugung/Wärmetransfer
		\end{enumerate}
	\item Druck: Steigt/Sinkt
	\item Wasserinventar sink grundsätzlich
	\item Kerntemperatur: Steigt bei Freilung, Sinkt bei herunterfallen, dann steigt wieder
\end{itemize}

\subsubsection{Containmentsumpf/Cavity Inventare}
\begin{enumerate}
	\item 14k s: ECCS Einspeisung \textrightarrow\ Wasserinventar steigt
	\item Keine Wasserkühlung \textrightarrow\ Wasserinventar sinkt
	\item 300k s: Containmentversagen: Massiver Verlust (Wasser verdampft)
	\item Pumpen stoppen (kein Wasser) \textrightarrow\ Cavity leert sich, wird trocken (MCCI), Schmilzt durch
\end{enumerate}

\subsubsection{Betonerosion}
\begin{itemize}[noitemsep]
	\item Bei trockener Cavity: Anstieg d. Erosion
	\item Unterseite zuerst komplett erodiert
	\item Unterseite unter kontinuierlichem Angriff
	\item Seiten erst ab trockener Cavity angegriffen
\end{itemize}

\subsubsection{Containmentdruck}
\begin{itemize}[noitemsep]
	\item Steigt an bis Containmentversagen, dann rapider Abfall
\end{itemize}

\subsubsection{Nuklide im Containment}
\begin{itemize}[noitemsep]
	\item Steigt anfangs
	\item Senkung viele Nuklide durch Spraysystem
	\item Senkung durch bei Containmentversagen
\end{itemize}

\subsection{ASTEC}
\begin{itemize}[noitemsep]
	\item Kontinuierlich upgedatet
	\item 2001: Release
	\item 2004: Neues MCCI-Modul
	\item 2005: Core-Quenching, PWR update
	\item 2011: Vorläufige Modelle for BWR/CANDU
	\item 2013: SA-Sequenzen für BWR/CANDU
\end{itemize}

\section{Schwere Unfälle}

\subsection{Frühe Unfälle - USA}
\begin{itemize}[noitemsep]
	\item 1954: Boiling Water Reactor Experiment No. 1 (BORAX-1)
	\item 1955: Experimental Breeder Reactor No. 1 (EBR-1)
\end{itemize}

\subsection{Chalk-River (1952)}
\begin{itemize}[noitemsep]
	\item Kühlung: Wasser, Moderator: Deuteriumoxid
	\item Plutonium/Radiopharmakaproduktion
	\item Bedienungsfehler \textrightarrow\ partielle Kernschmelze\\
		\textrightarrow\ \(H_2\)-Explosion \textrightarrow\ Kuppel zerstört \textrightarrow\ Spaltproduktfreisetzung
\end{itemize}

\subsection{SL-1 (1961)}

\subsubsection{Allgemein}
\begin{itemize}[noitemsep]
	\item Stationary Low Power Unit
	\item INEL
	\item BWR
	\item Prototyp f. transportable Energiequellen (arktische Radarstationen)
\end{itemize}

\subsubsection{Unfallverlauf}
\begin{enumerate}
	\item Reaktor war heruntergefahren
	\item Kontrollstäbe ausgehängt, Messgeräte rein
	\item Kontrollstäbe wieder einhängen (3 Mann), Reaktor hochfahren
	\item Kontrollstäbe händisch einhängen\\
		\textrightarrow\ Designfehler: Reaktor bei Entfernung d. zentralen Kontrollstabes kritisch
	\item Kontrollstab zu weit herausgezogen \textrightarrow\ Reaktor kritsch
	\item Wasser erhitzt \textrightarrow\ Dampf/Wasserhammer schleudern Kontrollstäbe heraus \textrightarrow\ Kernschmelze
\end{enumerate}

\subsubsection{Probleme}
\begin{itemize}[noitemsep]
	\item Designfehler: Kritikalität bei Entfernung des zentralen Kontrollstabes
	\item Wartungsfehler: Klemmende Kontrollstäbe nicht repariert
	\item Menschliches Versagen: Vorschriften nicht beachtet
\end{itemize}

\subsection{Windscale Pile No. 1 (1957)}

\subsubsection{Allgemein}
\begin{itemize}[noitemsep]
	\item Kühlung: Luft
	\item Moderator: Graphit
	\item Problem: Wigner-Energie
\end{itemize}

\subsubsection{Probleme}
\begin{itemize}[noitemsep]
	\item Falsche Instrumentenanzeige
	\item Unklarheit Reaktorzustand
\end{itemize}

\subsubsection{Unfallverlauf}
\begin{enumerate}
	\item Ausglühvorgang gestartet (CR ziehen)
	\item Falsche Instrumentenanzeige, erneuter Ausglühvorgang \textrightarrow\ T steigt stark
	\item Urankartusche platzt, oxidiert, T steigt \textrightarrow\ Feuer
	\item Crew nimmt an, \(\Delta T\) ist Ergebnis d. Ausglühens
	\item Da \(\Delta T\) hoch: Kühlung durch erhöhten Luftstrom\\
		\textrightarrow\ Mehr Oxidation, Spaltprodukte zu Filtern gebracht
	\item Erhöhte Aktivität registiert, Alarm ausgelöst
	\item Nach 1 Tag mit Wasser gelöscht, T war 1300°C
\end{enumerate}

\subsection{Three Mile Island-2 (1979)}

\subsubsection{Allgemein}
\begin{itemize}[noitemsep]
	\item PWR
	\item PORV (Pilot Operated Relief Valve)
		\begin{itemize}[noitemsep]
			\item Limitiert Druck im Kühlkreislauf
			\item Überdruck in RCDT
		\end{itemize}
	\item RCDT (Reactor Coolant Drain Tank)
		\begin{itemize}[noitemsep]
			\item Dampf kondensiert dort, zurück in Primärkreislauf
			\item Überströmventil \textrightarrow\ Kühlmittels in Containment
			\item Bei Überdruck: Berstscheibe bricht \textrightarrow\ Kühlmittel ins Containment
		\end{itemize}
\end{itemize}

\subsubsection{Probleme}
\begin{itemize}[noitemsep]
	\item Auslöser: Fehlfunktion Ventil Speiseleitung \textrightarrow\ SG trocken
	\item Signal verdeckt: Ventil zur sekundären Speisewasserpumpe geschlossen
	\item Fehlerhafte Anzeige: PORV bleibt offen (Fehlfunktion) \textrightarrow\ unbemerkter LOCA
	\item Operatoren ignorieren Start von HPCI/stoppen es, Annahme Primärsystem mit Wasser gefüllt
	\item Schwierigkeit Wiederherstellung des Kühlkreislaufs
\end{itemize}


\subsubsection{Unfallverlauf}
\begin{enumerate}
	\item Revisionsarbeiten (Kondensatreiniger reinigen)
	\item Hauptspeisewasserpumpe wg. Fehlfunktion inaktiv
	\item Sekundäre Speisewasserpumpe startet, aber Ventil geschlossen
		\begin{enumerate}[label = \textrightarrow]
			\item Loss of Feedwater Flow to Steamgenerator (SG)
			\item Austrocknung der SGs
			\item Keine Kühlung durch SGs
		\end{enumerate}
	\item Druckanstieg
		\begin{enumerate}[label = \textrightarrow]
			\item Druckhalteventil (PORV) bei 158 bar geöffnet
			\item Dampf in Abblasetank (RCDT)
			\item 3m: Überströmventil öffnet sich \textrightarrow\ Kühlmittel in Containment
			\item 15m: Überdruck \textrightarrow\ Berstscheibe bricht \textrightarrow\ Kühlmittel in Containment
		\end{enumerate}
	\item 9s: SCRAM
		\begin{enumerate}[label = \textrightarrow]
			\item CR werden eingefahren
			\item Druckanstieg gestoppt
			\item PORV sollte schließen, aber mechanischer Fehler
			\item Fehlerhafte Anzeige: PORV erscheint geschlossen, bleibt aber offen
			\item Small Break LOCA
		\end{enumerate}
	\item 2m: SGs trocken, Druck sinkt wegen PORV
		\begin{enumerate}[label = \textrightarrow]
			\item Wenn SGs einmal trocken \textrightarrow\ kein Wiederauffüllen möglich
			\item Start ECCS (HPCI)
			\item Ignoriert von Operatoren, ist immer wieder mal losgegangen
		\end{enumerate}
	\item 8m: Ventil sekundäre Speisewasserpumpe geöffnet \textrightarrow\ SGs bereits trocken
	\item 1h: Operatoren glauben, Primärsystem mit Wasser gefüllt
		\begin{enumerate}[label = \textrightarrow]
			\item ECCS ausgeschaltet
			\item Wasserniveau sinkt
			\item Kern freigelegt
			\item Temperatur steigt
			\item Freisetzung Spaltprodukte + \(H_2\)
			\item In Containment über Berstscheibe/Überströmventil
		\end{enumerate}
	\item 2h: Primärkreislaufpumpen ausgeschalten
		\begin{enumerate}[label = \textrightarrow]
			\item Wegen Vibrationen durch Kavitation (Dampf bereits im System)
			\item Beginn Austrocknung RPV
			\item Kerntemperatur steigt
			\item Kernfreilegung: Zircalloy + Dampf Reaktion
			\item Emission Wasserstoff, Fissionsprodukte
		\end{enumerate}
	\item 2h: PORV geschlossen \textrightarrow\ LOCA gestoppt
		\begin{enumerate}[label = \textrightarrow]
			\item ca 30\% d. Kerns geschmolzen \textrightarrow\ Operatoren keine Ahnung
			\item 700k Liter kontaminiertes Wasser in Containment
			\item Erstmals Strahlenalarm
		\end{enumerate}
	\item 3h: 50\% d. Kerns freigelegt
	\item 4-15h: Versuch Wiederherstellung Zirklulation / Drucksenkung
		\begin{enumerate}[label = \textrightarrow]
			\item HPCI starten, PORV öffnen
			\item nach einiger Zeit: Zirkulation startet, Druck sinkt
			\item Wasserstoff in Containment
			\item 9h: Wasserstoffexplosion
			\item 13h: PORV schließen, Hauptkühlmittelpumpen an
			\item 15h: Kühlung wiederhergestellt
		\end{enumerate}
\end{enumerate}

\subsubsection{Lehren}
\begin{itemize}[noitemsep]
	\item Schmelzvorgang beginnt bei Kontrollstäben
	\item Wasserdampf in unterem Plenum schützt Wand des RDB
	\item Containment effizente Barriere
		\begin{itemize}[noitemsep]
			\item Fast 100\% I/Cs in Wasser geblieben
			\item 50\% Edelgase in Containment
			\item 10\% Edelgase in Umwelt \textrightarrow\ Dosis \(<1 mSv\)
		\end{itemize}
\end{itemize}

\subsection{Tschernobyl (1986)}

\subsubsection{Probleme}
\begin{itemize}[noitemsep]
	\item Kernauslegung
		\begin{itemize}[noitemsep]
			\item Positiver Void-Koeffizent
			\item Positiver Abschalteffekt
			\item Unzureichende Wirksamkeit d. Abschalteeinrichtungen
		\end{itemize}
	\item Unzureichende Wahrnehmung v. Sicherheit
		\begin{itemize}[noitemsep]
			\item Keine unabhängige Sicherheitsbehörde
			\item Lieferschwierigkeiten f. Anlagenteile
		\end{itemize}
	\item Mängel im Versuchsprogramm
		\begin{itemize}[noitemsep]
			\item Ausarbeitung d. Elektroingenieur / keine Reaktortechniker beteiligt
			\item Keine Abstimmung m. Sicherheitspersonal
			\item Zeitliche Verschiebungen \textrightarrow\ Schichtwechsel
		\end{itemize}
	\item Verstöße gegen Betriebsvorschriften
		\begin{itemize}[noitemsep]
			\item Keine Erfahrung mit An/Abfahrprozessen (falsche Schicht)
			\item Unterschreitung Mindestwert Reaktivitätsreserve
			\item Unterschreitung d. vorgesehenen Werts f. thermischen Leistung
		\end{itemize}
	\item Ausbildungsmängel: keine Simulatoren, Sicherheitsanalysen
	\item Instrumentierungsmängel: keine Instrumentierung f. Zustände geringer Leistung
\end{itemize}


\subsubsection{Abschaltesystem}
\begin{itemize}[noitemsep]
	\item Steuerstäbe
		\begin{itemize}[noitemsep]
			\item Absorberteil (\(B_4C\), 6m)
			\item Verdrängerteil (Graphit, 4.5m)
				\begin{enumerate}[label = \textrightarrow]
					\item Graphit verdrängt n-absorbierendes Wasser
					\item positive Reaktivität
				\end{enumerate}
		\end{itemize}
	\item Einfahrzeit: 20s
\end{itemize}

\subsubsection{Versuchsprogramm}
\begin{itemize}[noitemsep]
	\item Nachweis d. Stromversorgung bei Reaktorabschaltung b. Totalausfall Stromversorgung
	\item Kann auslaufende Turbine genug Energie liefern, bis Notstromaggregate aktiv?
	\item Versuch bei reduzierter Leistung
	\item Planung ohne Reaktortechniker
\end{itemize}

\subsubsection{Unfallverlauf}
\begin{itemize}[noitemsep]
	\item 25.4 1:00: Herunterfahren d. Leistung (3000 \textrightarrow\ 1000 \(MW_{therm}\))
	\item Leistungsnachfrage \textrightarrow\ Herunterfahren bei 1600 MW unterbrochen
		\begin{enumerate}[label = \textrightarrow]
			\item Aufbau v. Xe-135
			\item Reaktivitätsreserve kleiner als zulässiger Wert
			\item \textbf{Sofortige Abschaltung hätte erfolgen müssen}
		\end{enumerate}
	\item 14:00: Notkühlsystem ausgeschalten (Testvorgabe)
	\item Wieder Leistungsnachfrage: Test unterbrochen
		\begin{enumerate}[label = \textrightarrow]
			\item \textbf{Notkühlsystem hätte wieder eingeschalten werden müssen}
		\end{enumerate}
	\item 23:00: Weiterführung d. Tests \textrightarrow\ Leistung wieder gesenkt
	\item 0:00: Schichtwechsel
	\item \textbf{Bedienungsfehler: Leistung sinkt auf 30MW (Mindestleistung 200MW)}
		\begin{enumerate}[label = \textrightarrow]
			\item Wg Xe-135: Steuerstäbe müssen zur Leistungserhöhung gezogen werden
			\item Erreicht werden 200MW
		\end{enumerate}
	\item 1:00: Turbineneinlassventile geschlossen
	\item 2 Hauptkühlmittelpumpen eingeschalten (Simulation Stromverbrauch)
		\begin{enumerate}[label = \textrightarrow]
			\item Dampfblasengehalt reduziert
			\item Reaktivität sinkt
			\item Kompensation: Weiteres Ausfahren d. Steuerstäbe
		\end{enumerate}
	\item 1:30: Testbeginn, Schließen Turbinenschnellschlussventile
		\begin{enumerate}[label = \textrightarrow]
			\item T steigt
			\item Leistungsanstieg
			\item Steuerstäbe eingefahren (zu langsam)
			\item n-Fluss steigt
			\item Xe-135 wird abgebaut
			\item Reaktor schaukelt sich hoch
		\end{enumerate}
	\item Notabschaltung d. Schichtleiter
		\begin{enumerate}[label = \textrightarrow]
			\item Alle Stäbe einfahren
			\item Graphitspitzen erhöhen Reaktivität
			\item Prompte Kritikalität
			\item Leistungsexkursion
			\item Chemische Explosion, Graphitfeuer
		\end{enumerate}
	\item nach 1 Tag: Block mit Material zugeschüttet

\end{itemize}

\subsubsection{Quellterm}
\begin{itemize}[noitemsep]
	\item 100\% Xenon
	\item 50\% Iod
	\item 30\% Cäsium
\end{itemize}

\subsection{Fukushima - 1 / Daiichi (2011)}

\subsubsection{Allgemein}
\begin{itemize}[noitemsep]
	\item Kontrollstäbe \(B_4C/Hf\)
	\item Containment in Betrieb mit \(N_2\) inertisiert
	\item Wetwell-Torus z. Druckentlastung (durch Kondensation)
\end{itemize}

\subsubsection{Probleme}
\begin{itemize}[noitemsep]
	\item Erdbeben: Alle Reaktoren heruntergefahren\\
		\textrightarrow\ \textbf{Loss of Offsite Power}
	\item Tsunami: zu Hoch
		\begin{itemize}[noitemsep]
			\item Dieselgeneratoren nicht operativ
			\item Generatoren selbst hoch genug
			\item Batterien/Verteiler zu tief
				\begin{enumerate}[label = \textrightarrow]
					\item Kein Notstrom
					\item \textbf{Station Blackout}
					\item Keine Pumpen operativ (auch ECCS)
				\end{enumerate}
		\end{itemize}
\end{itemize}

\subsubsection{Reaktorzustände}
\begin{itemize}[noitemsep]
	\item Fukushima I: Reaktor 4-6 nicht in Betrieb (Inspektion)
	\item 11 Reaktoren automatisch abgeschalten
\end{itemize}

\subsubsection{Isolation Condensor (IC)}
\begin{itemize}[noitemsep]
	\item Nur in Unit 1
	\item Passives Notfallsystem
	\item Normalerweise Ventil geschlossen
	\item Dampf aus RDB durch IC \textrightarrow\ Kondensiert \textrightarrow\ zurück zu RDB
	\item Wasserpool in IC verdampft langsam
\end{itemize}

\subsubsection{Unfallverlauf}
\begin{itemize}[noitemsep]
	\item Erdbeben \textrightarrow\ SCRAM
		\begin{itemize}[noitemsep]
			\item Reaktoren unversehrt
			\item Sicherheitsbehälter isoliert, Anlage in sicherem Zustand
			\item Zerfallswärme: nach SCRAM 6\%, 1d: 1\%, 5d: 0.5\%
		\end{itemize}
	\item Umschalten auf externe Stromversorgung, Schalten auf Diesel
	\item 2. Tsunami 17m Hoch
		\begin{itemize}[noitemsep]
			\item Dieselgeneratoren hoch genug, Verteiler jedoch nicht
			\item Meerwasser/Servicewasser-Einlassgebäude zerstört
			\item Station Blackout, nur noch teilweise Batterien
		\end{itemize}
	\item (Unit 2/3) 1 Dampfturbinenpumpe operativ, durch Reaktordampf getrieben
		\begin{itemize}[noitemsep]
			\item Dampf in Kondensationskammer (Wetwell) kondensiert
			\item Wasser aus Kondensationskammer in Reaktor gepumpt
			\item Benötigt Batterie, \(<100\text{°}C\)
			\item Keine Wärmeleitung aus Gebäude \textrightarrow\ Zeitlich limitiert, stoppt schließlich
		\end{itemize}
	\item Öffnung Druckentlastungsventile zu Kondensatorkammer
		\begin{itemize}[noitemsep]
			\item Wasserstand in RDB sinkt, Kern wird freigelegt
		\end{itemize}
	\item (Unit 1) IC Ventile geschlossen (kein Strom zum Öffnen)
	\item 5h: Druckbehälterversagen
	\item 7h: Containmentversagen \textrightarrow\ Aktivitätsanstieg
	\item 11h: Entscheidung f. Venting, erst nach 12 Stunden erfolgreich
	\item 1t: Feuerwasser aufgebraucht \textrightarrow\ Meerwasser
	\item 1t: \(H_2\) Explosion \textrightarrow\ Meerwasser/Stromversorgung gestört\\
		\textrightarrow\ nach 5 Stunden Meerwasserversorgung OK
	\item 1.5t: Meerwasserversorgung setzt erneut aus \textrightarrow\ erst nach 19h wiederhergestellt
	\item 2t: \(H_2\) Explosion in Unit 3
\end{itemize}

\subsubsection{Kernaspekte}
\begin{itemize}[noitemsep]
	\item nach Druckentlastungsventilöffnung
	\item Hüllrohrtemperaturen steigen
	\item 50\% Freilegung: noch kein signifikanter Kernschaden
	\item 66\% Freilegung: Hüllrohr 900°C, Ballooning/Bambooing
	\item Freisetzung Spaltprodukte
	\item 75\% Kernfreilegung: Hüllrohr 1200°C, Zr + \(H_2O\), zusätzliche Aufheizung
	\item Wasserstofferzeugung: 300kg bis (600kg (Block 1), 1000kg (Block 2/3))
	\item Wasserstoff gelangt in Kondensationskammer, durch Überdruckventile in Containment
	\item Aersole teilweise in Kondensationskammer gebunden, teilweise in Containment
\end{itemize}

\subsubsection{Aspekte d. Ventings}
\begin{itemize}[noitemsep]
	\item Entfernt Energie aus Containment (einzig verbliebener Weg)
	\item Reduziert Druck in Containment
	\item Freisetzung v. Aerosolen (I, Cs), gesamte Edelgase, Wasserstoff
	\item Gas in Reaktorgebäude abgeleitet
\end{itemize}

\subsubsection{Gründe f. Verzögerung d. Ventings}
\begin{itemize}[noitemsep]
	\item Kein Training unter Station Blackout Bedingungen
	\item Kein Manuals für bestimmte Aktionen vorhanden
	\item Ventilöffnung benötigt
		\begin{itemize}[noitemsep]
			\item Druckluft \textrightarrow\ keine mobilen Druckluftgeräte
			\item Strom \textrightarrow\ nicht Vorhanden
		\end{itemize}
	\item Arbeiten im Reaktorgebäude (\textrightarrow\ Ventile) verboten (Strahlung)
\end{itemize}

\subsubsection{Brennelementlagerbecken}
\begin{itemize}[noitemsep]
	\item Gesamter Kern v. Block 4 in Becken (Inspektion)
	\item Gefahr d. Austrocknung
		\begin{itemize}[noitemsep]
			\item Block 4: 10 Tage
			\item Restliche Blöcke: einige Wochen
			\item Ungewissheit: Leck wegen Erdeben?
		\end{itemize}
	\item Mögliche Konsequenzen
		\begin{itemize}[noitemsep]
			\item Brennelemente schmelzen an Luft
			\item Keine Rückhaltung v. Spaltprodukten
			\item Große Freisetzung möglich
		\end{itemize}
	\item Bergung Block 4: 2013-2014
\end{itemize}

\subsubsection{Status}
\begin{itemize}[noitemsep]
	\item Unit 1
		\begin{itemize}[noitemsep]
			\item Gebäudeschutz errichtet/abgebaut (Extraktion Brennelemente)
			\item Identifkation v. Trümmern
		\end{itemize}
	\item Unit 2
		\begin{itemize}[noitemsep]
			\item Blowout Panel geschlossen
			\item Hohe Strahlung im Gebäude
		\end{itemize}
	\item Unit 3
		\begin{itemize}[noitemsep]
			\item Trümmerentfernung v. Dach 2013 beendet
			\item Planung der Brennelementextraktion
			\item Hohe Strahlung
		\end{itemize}
	\item Unit 4
		\begin{itemize}[noitemsep]
			\item Brennelemente extrahiert
		\end{itemize}
\end{itemize}

	\end{document}
